\documentclass[10pt,pdf,hyperref={unicode}]{beamer}

\usepackage{lmodern}

% подключаем кириллицу 
\usepackage[T2A]{fontenc}
\usepackage[utf8]{inputenc}

\usepackage{amssymb,amsfonts,amsmath,mathtext,cite,enumerate,amsthm,mathenv} %подключаем нужные пакеты расширений


% тема оформления
\usetheme{CambridgeUS}

\makeatother
\setbeamertemplate{footline}
{
  \leavevmode%
  \hbox{%
  \begin{beamercolorbox}[wd=.3\paperwidth,ht=2.25ex,dp=1ex,center]{author in head/foot}%
    \usebeamerfont{author in head/foot}\insertshortauthor
  \end{beamercolorbox}%
  \begin{beamercolorbox}[wd=.7\paperwidth,ht=2.25ex,dp=1ex,center]{title in head/foot}%
    \usebeamerfont{title in head/foot}\insertshorttitle\hspace*{3em}
    \insertframenumber{} / \inserttotalframenumber\hspace*{1ex}
  \end{beamercolorbox}}%
  \vskip0pt%
}


% цветовая схема
\usecolortheme{seahorse}

\graphicspath{{images/}}%путь к рисункам

\title{Движение пассивного спутника в гравитационном поле Земли}
\author[Асланов Е.В.]{Студент: Асланов Евгений Владимирович, группа 1125 М 403\\ \and Научный руководитель: Асланов Владимир Степанович}
 \date[\today]{}

\begin{document}

% титульный слайд
\begin{frame}
\titlepage
\end{frame} 

\begin{frame}
\frametitle{Задачи}
	Исследование поведения нефункционирующего КА относительно центра масс на продолжительном отрезке времени.
	Подзадачи: 
	\begin{itemize}
		\item Установить влияние гравитационного момента на КА;
		\item Определить влияние малые возмущения моментов. ($M = M_G + M_1$) 
	\end{itemize}
\end{frame}

\begin{frame}
\frametitle{Уравнения движения центра масс}
	Запишем уравнения движения центра масс:
	
	\begin{equation*}
		\begin{cases}
			\frac{d\vec{r}}{dt} = \vec{V},\\
			\frac{d\vec{V}}{dt} = -\frac{1}{r^3} G \left(M + m\right) \vec{r},
		\end{cases}
	\end{equation*}
	где $G$ – гравитационная постоянная, $M$ – масса Земли, $m$ – масса КА.
\end{frame}

\begin{frame}
\frametitle{Уравнения движения КА относительно центра масс}
	\begin{equation*}
		\begin{cases}
			\frac{d\vec{\omega_x}}{dt} = \frac{M_x}{J_x} + \left(J_y - J_z\right) \frac{\omega_y \omega_z}{J_x},\\
			\frac{d\vec{\omega_y}}{dt} = \frac{M_y}{J_y} + \left(J_z - J_x\right) \frac{\omega_z \omega_x}{J_y},\\
			\frac{d\vec{\omega_z}}{dt} = \frac{M_z}{J_z} + \left(J_x - J_y\right) \frac{\omega_x \omega_y}{J_z},\\
			\omega_x = \frac{d\psi}{dt} \sin \theta \sin \phi + \frac{d\theta}{dt} \cos \phi, \\
			\omega_y = \frac{d\psi}{dt} \sin \theta \cos \phi - \frac{d\theta}{dt} \sin \phi, \\
			\omega_z = \frac{d\psi}{dt} \cos \theta + \frac{d\phi}{dt},
		\end{cases}
	\end{equation*}
	где $M_x$, $M_y$, $M_z$ – проекции момента сил тяготения, равные $M=\begin{pmatrix}\frac{3 GM}{R^3} (J_z - J_y) a_{32} a_{33} & \frac{3 GM}{R^3} (J_x - J_z) a_{33} a_{31} & \frac{3 GM}{R^3} (J_y - J_x) a_{31} a_{32}\end{pmatrix}$; $J_x$, $J_y$, $J_z$ – моменты инерции тела; $a_{31}$, $a_{32}$, $a_{33}$ – элементы матрицы поворотов.
\end{frame}
\begin{frame}
\frametitle{Вывод уравнений движения для динамически-симметрического КА в каноническом виде}
	Проекции абсолютной угловой скорости:
	
	\begin{equation}
	\label{eq:angle_v}
		\begin{cases}
			\omega_x = \dot{\psi} a_{31} + \dot{\theta} \cos \phi + \omega_0 a_{21}, \\
			\omega_y = \dot{\psi} a_{32} - \dot{\theta} \sin \phi + \omega_0 a_{22}, \\
			\omega_z = \dot{\psi} a_{33} + \dot{\phi} + \omega_0 a_{23}.
		\end{cases}
	\end{equation}
	
	Кинетическая энергия:
	
	\begin{equation}
	\label{eq:kinetic}
		T = \frac{1}{2} J_x (\omega_x^2 + \omega_y^2) + \frac{1}{2} J_z \omega_z^2.
	\end{equation}
	
	Потенциальная энергия:
	
	\begin{equation*}
		\Pi = \frac{3}{2} \omega_0^2 \left(J_z - J_x \right) \cos^2 \theta.
	\end{equation*}
\end{frame}

\begin{frame}
\frametitle{Функция Гамильтона}
	Функция Гамильтона:
	
	\begin{equation*}
		H = p_\phi \dot{\phi} + p_\psi \dot{\psi} + p_\theta \dot{\theta} - L,
	\end{equation*}
	где $L = T - \Pi$ – функция Лагранжа, $p_\phi = \frac{\partial{T}}{\partial{\dot{\phi}}}$, $p_\psi = \frac{\partial{T}}{\partial{\dot{\psi}}}$, $p_\theta = \frac{\partial{T}}{\partial{\dot{\theta}}}$ – обобщенные импульсы. 
	
	Подставив (\ref{eq:angle_v}) в (\ref{eq:kinetic}) и взяв частные производные, получим:
	
	\begin{equation*}
		\begin{cases}
			p_\phi = J_z \omega_z,\\
			p_\psi = J_x \left(\dot{\psi} \sin^2 \theta + \omega_0 \cos \psi \sin \theta \cos \theta \right) + J_z \omega_z \cos \theta, \\
			p_\theta = J_x \left(\dot{\theta} + \omega_0 \sin \psi \right).
		\end{cases}
	\end{equation*}
	
	И получим:
	\begin{equation*}
		\begin{cases}
			\dot{\theta} = \frac{p_\theta}{J_x} - \omega_0 \sin\psi, \\
			\dot{\psi} = \frac{p_\psi - J_z \omega_z \cos \theta}{J_x \sin^2 \theta} - \frac{\omega_0 \cos \psi}{\tan \theta},\\
			\dot{\phi} = \omega_z + \omega_0 \cos \psi \sin \theta - \dot{\psi} \cos \theta.
		\end{cases}
	\end{equation*}
\end{frame}

\begin{frame}
\frametitle{Функция Гамильтона}
	\begin{equation*}
		\begin{split}
			H &= \frac{1}{2} J_x \dot{\psi}^2 \sin^2 \theta + \frac{1}{2} J_x \dot{\theta}^2 + J_z \omega_z^2 +  J_z \omega_z \omega_0 \cos\psi \sin\theta + \\
			&+\frac{1}{2} \omega_0^2 J_x \cos^2\psi\sin^2\theta	+\frac{3}{2}  \omega_0^2 \left(J_z - J_x \right)\cos^2\theta
		\end{split}
	\end{equation*}	
	
	Подставив $\dot{\psi}$, $\dot{\theta}$ и $p_\phi = J_z \omega_z$ получим, что $H$ не зависит явно от $\phi$. Следовательно, $\phi$ циклическая и ей отвечает циклический интеграл
	\begin{equation*}
		p_\phi = const
	\end{equation*}
	или
	\begin{equation*}
		J_z \omega_z = J_z \omega_{z0}. 
	\end{equation*}
\end{frame}

\begin{frame}
\frametitle{Функция Гамильтона}
	Перейдем к безразмерным величинам:
	\begin{equation*}
		\begin {split}
			&\bar{p}_\psi = \frac{p_\psi}{J_x \omega_o}, \\
			&\bar{p}_\theta = \frac{p_\theta}{J_x \omega_o}, \\
			&\bar{p}_\phi = \frac{p_\phi}{J_x \omega_o}, \\
			&\tau = \omega_0 t.
		\end{split}
	\end{equation*}
	
	Итоговый вид:
	\begin{equation*}
		\begin{split}
			H &= \frac{\bar{p}_\psi^2}{2 \sin^2 \theta} + \frac{\bar{p}_\theta^2}{2} - \frac{\bar{p}_\psi \cos \psi}{\tan \theta} - \alpha \beta \bar{p}_\psi\frac{\cos \theta}{\sin^2 \theta} - \bar{p}_\theta \sin \psi + \alpha \beta \frac{\cos\psi}{\sin \theta} +\\
			&+ \frac{\alpha^2 \beta^2}{2 \tan^2 \theta} + \frac{3}{2} (\alpha - 1) \cos^2 \theta,
		\end{split}
	\end{equation*}
	где $\beta = \frac{\omega_{z0}}{\omega_0}$, $\alpha = \frac{J_z}{J_x}$.
\end{frame}

\begin{frame}
\frametitle{Уравнения движения в каноническом виде}
	\begin{equation*}
		\begin{cases}
			\frac{d\bar{p}_\psi}{d\tau} = - \frac{\partial H}{\partial \psi},\\
			\frac{d\psi}{d\tau} = \frac{\partial H}{\partial \bar{p}_\psi},\\
			\frac{d\bar{p}_\theta}{d\tau} = - \frac{\partial H}{\partial \theta},\\
			\frac{d\theta}{d\tau} = \frac{\partial H}{\partial \bar{p}_\theta}.
		\end{cases}
	\end{equation*}
	
	В явном виде:
	\begin{equation*}
		\begin{cases}
			\dot{\bar{p}}_\psi = - \left( \frac{\bar{p}_\psi}{\tan \theta} - \frac{\alpha\beta}{\sin \theta} \right) \sin \psi + \bar{p}_\theta \cos \psi, \\
			\dot{\bar{p}}_\theta = \frac{\bar{p}_\psi^2}{\sin^3 \theta} \cos \theta - \frac{ \bar{p}_\psi}{\sin^2 \theta} \cos \psi - \alpha \beta \bar{p}_\psi \frac{1 + \cos^2\theta}{\sin^3 \theta} + \alpha \beta \frac{\cos \psi}{\sin^2 \theta} \cos \theta + \\
			+ \frac{\alpha^2 \beta^2}{\tan \theta \sin^2 \theta} + 3 \left(\alpha -1\right) \cos \theta \sin \theta, \\
			\dot{\psi} = \frac{\bar{p}_\psi}{\sin^2 \theta} - \frac{\cos \psi}{\tan \theta} - \alpha\beta \frac{\cos\theta}{\sin^2 \theta}, \\
			\dot{\theta} = \bar{p}_\theta - \sin \psi.
		\end{cases}
	\end{equation*}
\end{frame}

\begin{frame}
\frametitle{Устойчивость стационарных положений}
	$\theta_0 = \frac{\pi}{2}$, $p_\psi^0 = 0$, $p_\theta^0 = \sin \psi_0$, $\cos \psi_0 = -\alpha\beta$
	
	\begin{columns}[onlytextwidth]
		\begin{column}{0.5\textwidth}
			\includegraphics[width=0.5\paperwidth]{stat_1_theta.png}
		\end{column}
		\begin{column}{0.5\textwidth}
			\includegraphics[width=0.5\paperwidth]{stat_1_psi.png}
		\end{column}
	\end{columns}
\end{frame}

\begin{frame}
\frametitle{Устойчивость стационарных положений}
	$\theta_0 = \frac{\pi}{2}$, $p_\psi^0 = 0$, $p_\theta^0 = 0$, $\sin \psi_0 = 0$
	
	\begin{columns}[onlytextwidth]
		\begin{column}{0.5\textwidth}
			\includegraphics[width=0.5\paperwidth]{stat_2_theta.png}
		\end{column}
		\begin{column}{0.5\textwidth}
			\includegraphics[width=0.5\paperwidth]{stat_2_psi.png}
		\end{column}
	\end{columns}
\end{frame}

\begin{frame}
\frametitle{Устойчивость стационарных положений}
	$\psi_0 = 0$, $p_\theta^0 = 0$, $p_\psi^0 = 3(\alpha - 1) \sin \theta_0 \cos \theta_0$, $\sin \theta_0 = \frac{\alpha\beta}{3 \alpha - 4}$
	
	\begin{columns}[onlytextwidth]
		\begin{column}{0.5\textwidth}
			\includegraphics[width=0.5\paperwidth]{stat_3_theta.png}
		\end{column}
		\begin{column}{0.5\textwidth}
			\includegraphics[width=0.5\paperwidth]{stat_3_psi.png}
		\end{column}
	\end{columns}
\end{frame}

\begin{frame}
\frametitle{Заключение}
	В данной работе повторены выводы Белецкого и записаны стационарные положения. Проверена численно их устойчивость.
	
	В дальнейшем планируется доработать модель для учета влияния малых возмущений, связанных с несферичностью Земли, воздействием аэродинамических сил, солнечным давлением и так далее.
\end{frame}

\begin{frame}
	\begin{center}
		Спасибо за внимание!
	\end{center}
\end{frame}

\begin{frame}
\frametitle{Список источников}    
	\begin{thebibliography}{10}    
		\beamertemplatebookbibitems
 		\bibitem{Markeev}
		А.П. Маркеев
		\newblock {\em Теоретическая механика}.
		\newblock Физматлит, 1990.    
		\beamertemplatebookbibitems
 		\bibitem{Belecki}
		В.В. Белецкий
		\newblock {\em Движение спутника относительно центра масс в гравитационном поле}.
		\newblock Издательство Московского университета, 1975.
	\end{thebibliography}
\end{frame}
\end{document}